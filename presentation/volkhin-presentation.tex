% !TEX TS-program = pdflatex
% !TEX encoding = UTF-8 Unicode

\documentclass{beamer}
\usepackage[T2A]{fontenc}
\usepackage[utf8]{inputenc}
\usetheme{Warsaw}
\usepackage{graphicx}
\usepackage[english,russian]{babel}
\usepackage{geometry}
\usepackage{amssymb}
\usepackage{pgf}
\usepackage[absolute,overlay]{textpos}

\title{Влияние микротрубочек на скорость образования стрессовых гранул}
\author{Артем Вольхин}
\institute{МФТИ}

\begin{document}


\begin{frame}
\titlepage
\end{frame}

%\begin{frame}{Содержание}
%\tableofcontents
%\end{frame}


\section{Стрессовые гранулы}
\begin{frame}{Стрессовые гранулы}
\begin{columns}
	\column{0.35\textwidth}
	\begin{figure}
	\pgfimage[width=1.1\columnwidth]{../pics/stressgranules}
	\end{figure}

	\column{0.75\textwidth}
	\begin{itemize}
	\item СГ~--- это скопления РНП в клетках
	\item Размер от 20 нм до 5 мкм
	\item Возникают при тепловом шоке, UV-облучении, окислительном стрессе
	\item Время образования 10--20 минут
	\item Двигаются в цитоплазме клетки
	\item При столкновениях сливаются в большие гранулы
	\end{itemize}
\end{columns}
\end{frame}

\section{Микротрубочки}
\begin{frame}{Микротрубочки}
	\begin{columns}
		\column{0.35\textwidth}
		\begin{figure}
		\pgfimage[width=\columnwidth]{../pics/microtubules}
		\end{figure}

		\column{0.75\textwidth}
		\begin{itemize}
		\item Микротрубочки~--- это белковые внутриклеточные структуры, входящие в состав цитоскелета
		\item Полые цилиндры диаметром 25 нм
		\item Образованы димерами тубулина
		\item Полярны~--- сборка на одном конце, разборка~--- на другом
		\item Перемещаются за счет элонгации
		\end{itemize}
	\end{columns}
\end{frame}


\section{Исследование стрессовых гранул}
\begin{frame}{Основные факты}
\begin{itemize}
\item СГ двигаются хаотически, 10\% неподвижны, 10\% двигаются направленно
\item Микрофилламенты препятствуют движению больших гранул
\item Наблюдается быстрое движение, не диффузия
\item СГ прикрепляются к микротрубочкам и двигаются вдоль них
\item Без МТ движутся медленнее и для образования больших СГ требуется больше времени
\end{itemize}
\end{frame}

\begin{frame}{Постановка задачи}
Основная задача при изучении СГ:
\begin{itemize}
\item Изучение числа, расположения и движения СГ
\item Изучение взаимодействия со структурой микротрубочек
%\item определение их скорости, типа движения и взаимодействия гранул между собой
\end{itemize}

Цель работы:
\begin{itemize}
\item Построить модель движения СГ и взаимодействия с микротрубочками
\item Проверить на модели теоретические результаты
\end{itemize}

Почему выбрано моделирование?
\end{frame}

\begin{frame}{Теоретические оценки}
Время между столкновениями РНП:
\begin{equation}
t_d \sim \frac{\eta}{K_B T C_{RNP}} \sim \frac{1}{4 \pi D r C_{RNP}}
\end{equation}

%Время до столкновения СГ с микротрубочкой:
%\begin{equation}
%t_m \sim \frac{1}{4 \pi D \left((a+r)^2 L \right)^{1/3} C_{RNP}}
%\end{equation}

{
\small
$\eta$~--- вязкость среды,
$r$ и $C_{RNP}$~--- радиус и концентрация изначальных РНП частиц,
$K_B T$~--- тепловая энергия,
$df$~--- фрактальная размерность, равная $\sim1.8$,
$a$~--- радиус микротрубочки,
L~--- ее длина,
$D$ и $D_s$~--- коэффициенты диффузии.
}
\end{frame}

\begin{frame}{Теоретические оценки}
Время до столкновения СГ с микротрубочкой:
\begin{equation}
t_m \sim \frac{1}{4 \pi D \left((a+r)^2 L \right)^{1/3} C_{RNP}}
\end{equation}

Время до столкновения с другой СГ на микротрубочке:
\begin{equation}
t_S \sim \frac{l_S^2}{D_S}
\end{equation}

{
\small
$r$ и $C_{RNP}$~--- радиус и концентрация изначальных РНП частиц, соответственно, $a$~--- радиус микротрубочки, L~--- ее длина, $D$ и $D_s$~--- коэффициенты диффузии, $l_S$~--- среднее расстояние между двумя частицами.
}
\end{frame}

\section{Модель}
\begin{frame}{Описание модели}
%	\setlength\leftmargin{0pt}
	\begin{columns}
		\column{0.4\textwidth}
		\begin{figure}
		\pgfimage[width=1.05\columnwidth]{../pics/1371544764}
		\end{figure}

		\column{0.8\textwidth}
		\begin{itemize}
		\item Замкнутный кубический объем со СГ и микротрубочками
		\item Броуновское движение частиц
		\item Слияние СГ при столкновении
		\item При столкновении с микротрубочками~--- присоединение и движение вдоль вдоль них
		\item Плотность сети микрофилламентов~--- порог подвижности
		\item 15 настраиваемых параметров
		\item Инструменты для исследования зависимости модели от поднабора параметров
		\end{itemize}
	\end{columns}
\end{frame}

%\begin{frame}{Скриншоты}
%\begin{figure}
%\pgfimage[height=0.8\textheight]{../pics/1371544764-eps-converted-to}
%\end{figure}
%\end{frame}

\section{Результаты}
\begin{frame}{Частота столкновения между гранулами}
\begin{figure}
\pgfimage[width=0.5\textwidth]{../results/P2pCollisionTimeOnDiffusionExperiment}%
\pgfimage[width=0.5\textwidth]{../results/P2pCollisionTimeOnConcetrationExperiment}
\end{figure}
\end{frame}

\begin{frame}{Частота столкновения СГ с микротрубочками}
\begin{figure}
\pgfimage[width=0.5\textwidth]{../results/P2mCollisionTimeOnLengthExperiment}%
\pgfimage[width=0.5\textwidth]{../results/P2mCollisionTimeOnRadiusExperiment}
\end{figure}
\end{frame}

\begin{frame}
\begin{figure}
\pgfimage[width=0.5\textwidth]{../results/CreatingLargeGranulesExperiment}%
\pgfimage[width=0.5\textwidth]{../results/P2pCollisionTimeOnVolumeExperiment}
\end{figure}
Зависимость времени образования СГ от ее размера (слева) и зависимость времени между столкновения СГ от размера среды (справа).
\end{frame}

\begin{frame}{Планы на будущее}
\begin{itemize}
\item В дальнейшем планируется повышение достоверности модели (более реалистичная сеть микротрубочек, СГ с изначально разными параметрами и другое)
\item Исследование и поиск новых зависимостей
\end{itemize}
\end{frame}

\begin{frame}
\centering\Huge
Спасибо за внимание!
\end{frame}

\end{document}